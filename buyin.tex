\section{ZTF2 Collaboration Buy In}

While a successful peculiar velocity survey is possible from public ZTF2 alerts,
there are benefits to joining the private survey, including having the power to design both public and private surveys and
access to data that gives better early classification and SN distances.
ZTF2 requires buy-in: here are some options for LBL

\subsection{Through DESI}

DESI could try to negotiate buy-in through in-kind contributions of galaxy redshift catalogs and observations.  DESI members, including interested
LBL and non-LBL collaborators, would gain access to the ZTF2 collaboration.

DESI would offer some of the following
\begin{itemize}
\item Early access to spectra of transient host-galaxies taken as part of its surveys, the BGS in particular.
\item Prioritize observations of those DESI  targets that happen to be of interest to ZTF2.
\item Secondary science fiber overrides in DESI pointings.
\item DESI pointing overrides for objects in fields with no planned near-term observations.
\item Galactic time allocation, in anticipation that at some seasons DESI will not have many extra-galactic pointings.
\item Redshifts of ``all'' transient hosts at the conclusion of ZTF2
\item Joint DESI-ZTF2 density plus peculiar-velocity analysis.
\end{itemize}

Toward the end of the DESI survey there will be an opportunity for pilot studies for an extension or next-generation DESI.  
We can develop and advocate such a study beneficial to ZTF2.

\subsection{Through NERSC}
Current plans call for IPAC to perform ZTF2 data management. There is a constituency of current ZTF stakeholders who
would like for NERSC to take on this responsibility for ZTF2, as IPAC has not delivered desired photometric accuracies.

Peter, beyond running code at NERSC, what more could LBL commit to? 


\subsection{LSST In-Kind Contribution}
The new ``open data'' model for LSST has non-US scientists looking for in-kind contributions to buy into
LSST.  International scientists are informally asking LBL folks  whether  ZTF2 could be a contribution.
Through the process initiated by DESC, we plan to advocate for ZTF2-LSST collaboration.
Any such dealing would be done at the funding agency level with input from the LSST Project,
it is doubtful that LBL could play a leadership role in such negotiations.
