

\section{Conclusions}

%SNe~Ia are already powerful probes of the homogeneous cosmological expansion of the Universe.  
In the next decade,
the high number of SN discoveries together with improved precision in their distance precisions will make $z<0.2$ SNe~Ia, more so than
galaxies,  powerful probes of gravity through their effect  on the growth of structure.  No other probe of growth of structure or tracer of peculiar velocity can alone provide comparable precision on  $\gamma$ in the next decade.
At low redshift, the RSD measurement is quickly sample variance limited (as are the planned DESI BGS and 4MOST surveys) making peculiar velocities the only 
precision probe of $fD$.
TAIPAN and a TAIPAN-like DESI BGS will be able to measure FP distances for nearly all usable nearby galaxies, so at low-$z$ the Fundamental Plane peculiar-velocity
technique will  saturate at a level that is not competitive with a  2-year SN survey.
