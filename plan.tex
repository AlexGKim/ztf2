\section{Plan for a Peculiar Velocity Program}
The projections for measuring $\gamma$ with ZTF2 and LSST SN~Ia discoveries
show the power of peculiar velocities surveys at low redshift.
This  tracer provides an unmatched  new window with which to test gravity and the source of the accelerating expansion of the Universe.
We propose the following course of research for the upcoming decade.  The general theme is that we will use public discoveries and data
from ZTF-II and LSST, and supplement them with the additional data  described
in \S\ref{science:sec} necessary to make a  peculiar velocity measurement.

\subsection{SNfactory}
We will perform a PV analysis using the precise and accurate distances of supernovae obtained by the SNFactory.
Kim has developed a SNFactory/ZTF peculiar-velocity analysis framework with Graziani.  Its distinguishing features is that it is likelihood-based
including both density and peculiar velocities.  The complexity of including fitting the underlying mass density field in the model
is addressed through Hamiltonian Monte Carlo.  Analytic expressions for the partial derivatives of the likelihood are coded,
avoiding the computational limitations of {\it autodiff} in STAN.  A end-to-end implementation is complete
and a draft methodologies paper is expected within a week.


This work is done as part of the SNfactory and specifically with French IN2P3 colleagues.

\subsection{DESI as TAIPAN-North}

DESI has the potential to perform a compelling Peculiar Velocity Survey.  Its northern hemisphere coverage complements 
the TAIPAN and WALLABY southern surveys.   Its technical capabilities far exceed those of TAIPAN.  While ideally a PV-optimized
survey with the instrument would be preferred, we are examining how much can be done within the Bright Galaxy Survey (BGS).

The BGS targets objects to a fainter magnitude limit than TAIPAN.  However, the DESI $S/N$ and single visits are a point of concern.
Indeed, previous Fundamental Plane surveys have found that per-observation systematic errors in measured velocity dispersion are the limiting
source of error.  For this reason, TAIPAN revisits each Fundamental Plane galaxy several times to $\sqrt{N}$-suppress this error.
Kim is working with a PV subgroup within DESI to determine the statistical error in velocity dispersion from BGS observations.
We are would like multiple visits of the same source during SV to quantify any extra variance that appears in
the dispersion measurement.

While the tall pole is the velocity-dispersion measurement, the
imaging component is well in hand. Kim has worked with DESI-colleague Parkinson to find that DESI Legacy Imaging Surveys DR8  performs
as well as currently-used releases in measuring galaxy surface brightnesses and angular sizes, and have identified new Tractor
per-band
pixel-level reductions that could further increase the performance.

This work is done within the DESI collaboration, mostly with Howlett and Parkinson.  Blake and Davis are interested as well.

\subsection{DESI getting redshifts for SN~Ia PV with ZTF, ZTF-II}
The redshift precision obtained by ZTF's SEDMachine is poor enough
to adversely affect peculiar velocity measurements.  In addition, pre-existing host-galaxy redshifts
 aid in the classification of transients.  As such, we have been in working with our DESY/Humboldt University
colleagues on an external collaboration agreement with DESI, who would provide a precision redshift of ZTF transient hosts,
the vast majority are already targets of the BGS survey.   DESI members would gain access to ZTF Partnership
SN~Ia light curves.

This work has been done through the DESI Time Domain Working Group.
A collaboration agreement between DESI and ZTF-II is also possible though it won't be initiated until there is a ZTF-II Partnership.

\subsection{SN~Ia Follow-up Network of ZTF-II, LSST, and Other Sources of SN Discovery}
In the future, ZTF-II and LSST will provide public nearby SN discoveries and photometry and in the case of ZTF-II some spectroscopy.
The ability to determine accurate from these public data varies.  ZTF-II public $gr$ photometry itself will not give precise distances.
SEDMachine can provide classifications and perhaps precise distances, but only a small fraction of SNe will have public data.
ZTF-II lacks precision redshifts .
The LSST WFD survey observing strategy is yet to be specified.  All strategies considered give complete discovery out to $z=0.3$, but
they vary widely on their ability to yield accurate distances or early classification.  LSST provides no spectroscopy. Either way, none of these public data
(nor any of the ZTF-II private) will provide the precision that spectrophotometry can.  For this reason, LBL's primary interest
is in developing a follow-up network covering the sky of both northern and southern hemisphere searches.

There are several classes of follow-up we have identified:
\begin{itemize}
\item Spectrophotometry around peak brightness.  Spectrophotometry is expected to give 0.08~mag distance modulus uncertainties compared
to the 0.12~mag uncertainties of a private ZTF or good-survey-strategy WFD LSST photometry.  Spectrophotometry increases the power
of a single supernova by $\times 2.25$ while also getting a host redshift.
\begin{itemize}
\item SN Ia Classification -- SNIFS can supplement SED Machine to go toward 100\% SN~Ia classifications
while going deeper than the nominal $18.5$~mag limit of SED Machine.
\item Host-galaxy redshift.
\item SN Ia Distance -- SNIFS has already been used to standardize SNfactory SNe~Ia magnitudes to $\sigma_M=0.08$~mag.  SNIFS follow-up
of ZTF-II and LSST will be
designed to obtain this precision.  This SN subset  will have relatively smaller peculiar-velocity uncertainties relative to 
those with only photometry.  The SNIFS IFU provides local host-galaxy properties, which may also improve SN distance precisions.
\end{itemize}

In the above 
SNIFS at the UH-88 is called out as an existing instrument that could do the job and Greg is communicating with UH about
a robotized upgrade to follow-up ZTF discoveries.  The University of Hawaii must allocate time and resources into the program.  There is already UH expertise in supernovae and peculiar velocities and an existing relationship
with LBL.

We estimate that 2 or 3 dedicated 2m telescopes instrumented with similar IFUs could provide
complete followup of $z<0.09$ discoveries in concurrent northern and southern searches.
We submitted a proposal to instrument the CAHA 2m with
an IFU to follow ZTF discoveries.  IN2P3 Colleagues are interested in installing a MUSE clone on the VST.

LBL should actively be
\begin{itemize}
\item Developing its ability to design and build IFU spectrographs.  We can leverage the expertise used to design the DESI spectrographs.
There are several IFU technologies.  LBL/SSL experience with fibers makes a lens-array fiber-bundle IFU a natural technology to develop
internally.
Alternatively, Marseille and Lyon colleagues have expertise in the other IFU technologies, image slicer and lenslet arrays,
\item We need to identify available telescopes that we can instrument with  IFU spectrographs
built by LBL and collaborators.
\end{itemize}
\item Optical imaging.  LSST may choose a WFD strategy that while discovering SNe~Ia, provide so little data on them
as to render them useless.  Even if LSST does have a good observing strategy, the closest SNe~Ia that we are interested in
will saturate the LSST Camera. 

The needed imaging depends on the selected WDF strategy and the amount of spectrophotometry
we get.  An extreme case occurs when early classification is not possible with LSST.  Then a
``ZTF-South'' instrument capable of SN discovery early in their evolution and not saturating is needed.
Greg and Peter think that a 1m-class Schmidts would be necessary.  We have not identified such a telescope
that fits the bill.

If only a small number of supplemental observations are necessary of low surface density targets, a smaller FOV camera would suffice.

\item NIR data, like spectrophotometry, have been used to get $\sigma_M=0.08$~mag. UH and Carnegie colleagues have been interested
in this.  Perhaps LBL has a spare SNAP HgCdTe sitting around that can be used for this.
\item DESI can provide the following components to supplement surveys:
\begin{itemize}
\item Discovery Screening -- Provides redshifts for probable host galaxies of new transients, for use in early classification.  A host galaxy may
already have a DESI redshift, may be a BGS target without a redshift but whose observation could be prioritized,  or non-DESI target for which we
make a secondary-target fiber allocation.
\item SN Ia Classification -- Spectroscopy of active transients  through secondary-target fiber allocations within
DESI survey pointings. There
will be $\sim 1$ active $r<21.5$ SN~Ia in every three DESI pointings.  Coordinating DESI pointings such that ``every'' pointing contains
a  ZTF2 discovery can significantly increase the number of classifications, relative to the random (from the transient perspective) default DESI pointings.
Triggered observations of non-DESI pointings is possible, though does not take advantage of DESI multiplexing.
\item Host-galaxy redshift -- Mopping up missing host-galaxy redshifts
can be done efficiently with a single sweep of DESI's MOS.
\end{itemize}
 The $R>2000$ resolution
provides sufficiently precise redshifts so as to make their uncertainties negligible in the $\gamma$ error budget.

\end{itemize}

LBL, through the project, would support a broad community interested in peculiar velocities. 
There are interested groups at the  University of Hawaii, the University of Michigan, the University of Pittsburgh, the University of Rochester, Yale University,  Brookhaven National Laboratory,
the Carnegie Observatories, several IN2P3 labs, Humboldt University, the University of Toronto,  the University of Queensland, African Institute for Mathematical
Sciences.
There is expressed interest from South Africa and Australia in using their resources for peculiar-velocity follow-up observations.

\subsection{ZTF-II Partnership}

ZTF is soon concluding.  LBL plans to use public ZTF-II discoveries as the source of SNe for a near-term peculiar-velocity program.
There are advantages to being members of the partnership.
for the following reasons:
\begin{itemize}
\item Being a partner brings a voice as to how partnership ZTF and SNM time are to be used.  
A peculiar-velocity survey is already being touted as a primary science driver, and we can push for it.
\item It would be ready to start Fall 2020.
\item It is cheap, with a cost ranging from zero to access public data, to an amount (\$200k/yr) smaller
than building a new facility.
\item ZTF2 includes SEDMachine for classification of $m<18.5$~mag  transients.
\item It is in the Northern Hemisphere, which complements and is not superseded by LSST. 
Even after the nominal 3-year survey and simultaneous with LSST, the facility remains important for peculiar-velocity studies.
\item It is anticipated that the public plus private collaboration surveys can be designed to generate distance
precisions of $\sigma_M =0.12$~mag, which allow good velocity measurements with SNe~Ia.  (Additional follow-up can lower this uncertainty further.)
\item The limiting redshift $z_{\text{max}} =0.09$ is sufficiently deep  to have a scientifically interesting result $\sigma_\gamma < 0.053$.
There are other SN searches that do not achieve this depth.
\end{itemize}

ZTF2 SN~Ia discoveries and pubic/partnership data can be supplemented  with data from other facilities, as described above, to form a complete PV program.  
The active SN spectrophotometric and NIR follow-up provide significant distances estimates over
what ZTF2 photometry can do alone.

I think (?!) the expectation would be that the science would be run through the partnership.  Value-added data would be shared with other ZTF-II
partners interested in the same science.

%ZTF2 will have public, private collaboration, and CalTech surveys.    In ZTF,
%the private time was used to survey in the $i$-band (supplementing the $g$ and $r$-bands of the public survey), that turns out to be useful in transient classification and SN~Ia distance determination.
%Nevertheless, we should monitor whether the public data is sufficient for our needs.  It could be that we do not need to buy into ZTF2 in order to have
%a ZTF2-discovery peculiar velocity survey.  (Input from current ZTF folks should be solicited.)

Option A: Do not join the partnership and use ZTF-II public data.
ZTF-II public should provide discoveries and classifications of $z<0.09$ efficiently.  SEDMachine spectra for a
fraction of the bright objects.  Public pixels will be available in two months.
The pro is that the discoveries are free.
Cons are  that to be competitive we nonetheless need to get more external data to make up for the lack of (competing) partnership data
we don't get.  
Either way precise redshifts have to come from somewhere.


Option B: Join the ZTF-II partnership.
Private buy-in probably gives slightly more solid-angle, $i$ photometry, and more SEDMachine spectra.  With the third band distance precisions are expected
to be usable for PV, though not as good as with spectra at maximum.  This is a low-risk option to give reasonable science.
It gives a say as to how to steer partnership time, though there is no guarantee that
you can generate a majority.

Option C: Side-door entry through DESI.  Early DESI host redshifts and ZTF-II discovery follow-up of interest to ZTF-II.

\subsection{LSST +}
The DESC SN~Ia Working Group is interested in peculiar velocity science.  There is an official peculiar-velocity project.  A peculiar-velocity
metric was included in the DESC response to the Project call for white papers on observing strategies.
Informal meetings have been held by DESC members.
Building a community via DESC is a natural way to socialize the need for a follow-up work network through DOE and perhaps
IN2P3.

\begin{comment}
\subsection{A New Project}
The scope of ZTF2 and the coordination of follow-up of LSST discoveries extend beyond the confines
of current DOE projects.  The recommendations
of the  Small Projects Portfolio  by the  Cosmic Visions Dark Energy Working Group
provides an path by which LBL could lead an international collaboration,
supported by the Office of Science, in the study of Peculiar Velocities.

The new peculiar velocity project would focus on two topics: the use of DESI (and future spectroscopic surveys
such as DESI2) for measuring distances of
fundamental plane galaxies; the mobilization of follow-up resources and data management that are required or enhance 
the probative power of transient discoveries by ZTF2/LSST.  Ideas being discussed for the latter include
refurbishment and use of the
UH-88 + SNIFS, DESI, 4MOST, a proposed French spectrograph mounted at ESO,
a network of identical spectrographs (e.g.\ the DESI design) distributed around the world.
There is expressed interest from South Africa and Australia in using their resources for peculiar-velocity follow-up observations.
\end{comment}

