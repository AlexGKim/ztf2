\section{Plan for a Peculiar Velocity Program}
The projections for measuring $\gamma$ with ZTF2 and LSST SN~Ia discoveries
show the power of peculiar velocities surveys at low redshift.
This  tracer provides an unmatched  new window with which to test gravity and the source of the accelerating expansion of the Universe.
We propose the following course of research for the upcoming decade.

\subsection{Present}
The current goals are to provide a proof of concept of a SN peculiar velocity survey while developing domain
expertise and pipelines that can be used in future experiments.

Kim has developed a SNFactory/ZTF peculiar-velocity analysis framework with a postdoc.  Its distinguishing features is that it is likelihood-based
including both density and peculiar velocities.  The complexity of including fitting the underlying mass density field in the model
is addressed through Hamiltonian Monte Carlo.  Analytic expressions for the partial derivatives of the likelihood are coded,
avoiding the computational limitations of {\it autodiff} in STAN.  A end-to-end implementation is complete, validation of it is ongoing.

The first application o the analysis pipeline will be for SNFactory supernovae.  A next analysis will be of ZTF-discovered 
SNe when those data are ready.  The plan is for a  subset of the required data, precision redshifts of SN host galaxies, to be provided
by DESI.  Additional DESI data contributions are being discussed.

\subsection{Near Term: DESI as TAIPAN-North}

There has been talk of using BGS for a peculiar-velocity survey.  There is some question as to how deep the BGS achieves.  This should be
dug into.

\subsection{Near Term: ZTF2 + DESI}
A near-term peculiar velocity program can already provide the most competitive measurements of $\gamma$ at low redshift.  Engaging in science
now positions LBL for domain leadership in the LSST era.  For its near-term peculiar-velocity program, we advocate a survey using SN~Ia discoveries from ZTF2,
rather than other possible surveys, for the following reasons:
\begin{itemize}
\item A peculiar-velocity survey is already being touted as a primary science driver. As such,
the observing strategy should accommodate our needs.
\item It would be ready to start at the end of ZTF in 2021.
\item It is cheap, with a cost ranging from zero to access public data, to an amount (\$300k?) smaller
than building a new facility.
\item ZTF2 includes SEDMachine for classification of $m<18.5$~mag  transients.
\item It is in the Northern Hemisphere, which complements and is not superseded by LSST. 
Even after the nominal 3-year survey and simultaneous with LSST, the facility remains important for peculiar-velocity studies.
\item It is anticipated that the public plus private collaboration surveys can be designed to generate distance
precisions of $\sigma_M =0.12$~mag, which allow good velocity measurements with SNe~Ia.  (Additional follow-up can lower this uncertainty further.)
\item The limiting redshift $z_{\text{max}} =0.09$ is sufficiently deep  to have a scientifically interesting result $\sigma_\gamma < 0.053$.
There are other SN searches that do not achieve this depth.
\end{itemize}

ZTF2 SN~Ia discoveries are combined with data from other facilities to form a complete PV program.  We propose:
\begin{itemize}
\item DESI will provide the following components of the survey:
\begin{itemize}
\item Discovery Screening -- Provides redshifts for probable host galaxies of new transients, for use in early classification.  A host galaxy may
already have a DESI redshift, may be a BGS target without a redshift but whose observation could be prioritized,  or non-DESI target for which we
make a secondary-target fiber allocation.
\item SN Ia Classification -- Spectroscopy of active transients  through secondary-target fiber allocations within
DESI survey pointings. There
will be $\sim 1$ active $r<21.5$ SN~Ia in every three DESI pointings.  Coordinating DESI pointings such that ``every'' pointing contains
a  ZTF2 discovery can significantly increase the number of classifications, relative to the random (from the transient perspective) default DESI pointings.
Triggered observations of non-DESI pointings is possible, though does not take advantage of DESI multiplexing.
\item Host-galaxy redshift -- Some precise galaxy redshifts may not be available at the end of the ZTF2 survey.  Mopping up of ZTF2 host-galaxy redshifts
can be done efficiently with a single sweep of DESI's MOS.
\end{itemize}
 The $R>2000$ resolution
provides sufficiently precise redshifts so as to make their uncertainties negligible in the $\gamma$ error budget.
Its BGS targets  will host a large fraction of ZTF2-discovered SNe~Ia.
\item SNIFS at the UH-88 is used to spectroscopically observe a subset of active likely-SN~Ia transients.  SNIFS provides simultaneously
\begin{itemize}
\item SN Ia Classification -- SNIFS can supplement SED Machine to go toward 100\% SN~Ia classifications
while going deeper than the nominal $18.5$~mag limit of SED Machine.
\item Host-galaxy redshift.
\item SN Ia Distance -- SNIFS has already been used to standardize SNe~Ia magnitudes to $\sigma_M=0.08$~mag.  SNIFS observations will be
designed to obtain this precision.  This SN subset  will have relatively smaller peculiar-velocity uncertainties relative to 
those with only ZTF2 photometry.  The SNIFS IFU provides local host-galaxy properties, which may also improve SN distance precisions.
\end{itemize}
The University of Hawaii must allocate time and resources into the program.  There is already UH expertise in supernovae and peculiar velocities and an existing relationship
with LBL.
\item NIR -- Something through UH?
\begin{itemize}
\item SN Ia Distance -- NIR observations are designed to get $\sigma_M=0.08$~mag.  These SNe will be more sensitive probes of velocity
than those with only ZTF2 photometry.
\end{itemize}
\end{itemize}
The active SN spectrophotometric and NIR follow-up provide significant distances estimates over
what ZTF2 photometry can do alone.




ZTF2 will have public, private collaboration, and CalTech surveys.    In ZTF,
the private time was used to survey in the $i$-band (supplementing the $g$ and $r$-bands of the public survey), that turns out to be useful in transient classification and SN~Ia distance determination.
Nevertheless, we should monitor whether the public data is sufficient for our needs.  It could be that we do not need to buy into ZTF2 in order to have
a ZTF2-discovery peculiar velocity survey.  (Input from current ZTF folks should be solicited.)

Plan A: Use public data only.  ZTF-II public should provide discoveries and classifications of $z<0.09$ efficiently.  It will provide SEDMachine spectra for a
fraction of the bright objects.  Its two-band photometry will not provide good distances.  SNIFS can follow-up a subset of these at peak brightness.
Good PV's can be obtained from these.  Follow-up in other bands supplementing ZTF data can provide distances.

Private buy-in probably gives slightly more solid-angle, $i$ photometry, and more SEDMachine spectra.  With the third band distance precisions are expected
to be usable for PV, though not as good as with spectra at maximum.  It gives a say as to how to steer partnership time, though there is no guarantee that
you can generate a majority.

The overall assessment is that buy-in would be nice as a second but not first priority.

\subsection{Long Term: LSST +}
The DESC SN~Ia Working Group is interested in peculiar velocity science.  There is an official peculiar-velocity project.  A peculiar-velocity
metric was included in the DESC response to the Project call for white papers on observing strategies.
Informal meetings have been held by DESC members.

The nominal LSST observing strategy will efficiently discover SNe~Ia at $z<0.3$.  However, its poor temporal sampling will not provide accurate
distances nor early classification of SNe~Ia near maximum light.  Our French colleagues have identified a different strategy that will provide
decent sampling that give accurate distances and classifications.  LBL's approach varies depends on whether the Project adopts an acceptable
WDF observing strategy.  A systemic problem is the saturation of nearby SNe around peak brightness.

If it does not, we can either run a ``ZTF South'' that is coordinated with LSST, or the surface density is low enough that we could take
a targeted approach.  The combined photometry can be used to get distances.  This would be supplemented with spectroscopy as described below.

If LSST WFD does get respectable distances and classifications, our primary interest is value added information from spectroscopy.

\subsection{A New Project}
The scope of ZTF2 and the coordination of follow-up of LSST discoveries extend beyond the confines
of current DOE projects.  The recommendations
of the  Small Projects Portfolio  by the  Cosmic Visions Dark Energy Working Group
provides an path by which LBL could lead an international collaboration,
supported by the Office of Science, in the study of Peculiar Velocities.

The new peculiar velocity project would focus on two topics: the use of DESI (and future spectroscopic surveys
such as DESI2) for measuring distances of
fundamental plane galaxies; the mobilization of follow-up resources and data management that are required or enhance 
the probative power of transient discoveries by ZTF2/LSST.  Ideas being discussed for the latter include
refurbishment and use of the
UH-88 + SNIFS, DESI, 4MOST, a proposed French spectrograph mounted at ESO,
a network of identical spectrographs (e.g.\ the DESI design) distributed around the world.
There is expressed interest from South Africa and Australia in using their resources for peculiar-velocity follow-up observations.

LBL, through the project, would support a broad community interested in peculiar velocities. 
There are interested groups at the  University of Hawaii, the University of Michigan, the University of Pittsburgh, the University of Rochester, Yale University,  Brookhaven National Laboratory,
the Carnegie Observatories, several IN2P3 labs, Humboldt University, the University of Toronto,  the University of Queensland, African Institute for Mathematical
Sciences.