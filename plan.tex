\section{Plan for a Peculiar Velocity Program}
The projections for measuring $\gamma$ with ZTF2 and LSST SN~Ia discoveries
show the power of peculiar velocities surveys at low redshift.
This  tracer provides an unmatched  new window with which to test gravity and the source of the accelerating expansion of the Universe.
We propose the following course of research for the upcoming decade.

\subsection{SNfactory}
Kim has developed a SNFactory/ZTF peculiar-velocity analysis framework with Graziani.  Its distinguishing features is that it is likelihood-based
including both density and peculiar velocities.  The complexity of including fitting the underlying mass density field in the model
is addressed through Hamiltonian Monte Carlo.  Analytic expressions for the partial derivatives of the likelihood are coded,
avoiding the computational limitations of {\it autodiff} in STAN.  A end-to-end implementation is complete
and a draft methodologies paper is expected within a week.


This work is done in collaboration with French IN2P3 colleagues.

\subsection{DESI as TAIPAN-North}

DESI offers the possibility of a compelling Peculiar Velocity Survey,  DESI's northern hemisphere coverage complements 
the TAIPAN and WALLABY souther surveys.   Its technical capabilities far exceed those of TAIPAN.  While a PV-specific
survey is feasible with the instrument, we are examining how much can be done within the Bright Galaxy Survey (BGS).

The BGS targets objects to a fainter magnitude limit than TAIPAN.  However, the DESI $S/N$ and single visits are a point of concern.
Indeed, previous surveys have found that per-observation systematic errors in measured velocity dispersion are the limiting
source of error.  For this reason, TAIPAN revisits each Fundamental Plane galaxy several times to $\sqrt{N}$-reduce this error.
Kim is working with a PV subgroup within DESI to determine the statistical error in velocity dispersion from BGS observations.
We are requesting multiple visits of the same source during SV to quantify any extra variance that appears in
the dispersion measurement.

While the tall pole is in the dispersion measurement, the
imaging component is well in hand. Kim has worked with DESI-colleague Parkinson to find that DESI Legacy Imaging Surveys DR8  perform
as well as previous releases to measuring galaxy surface brightnesses and angular sizes, and have identified new
pixel-level reductions that could further increase the performance.

This work is done within the DESI collaboration, mostly with Howlett and Parkinson.  Blake and Davis are interested as well.

\subsection{DESI getting redshifts for SN~Ia PV with ZTF, ZTF-II}
The redshift precision obtained by ZTF's SEDMachine is poor enough
to adversely affect peculiar velocity measurements.  As such, we have been in working with our DESY/Humboldt University
colleagues on an external collaboration agreement with DESI who would provide a precision redshift of ZTF transient hosts,
the vast majority are already targets of the BGS survey. 

This work has been done through the DESI Time Domain Working Group.

A collaboration agreement between DESI and ZTF-II is also possible though won't be initiated until there is a ZTF-II Partnership.

\subsection{SN~Ia Follow-up Network}
In the future, ZTF-II and LSST will provide public nearby SN discoveries and photometry and in the case of ZTF-II some spectroscopy.
The ability to determine accurate from these public data varies.  ZTF-II public $gr$ photometry only will not give precise distances.
It is possible that SEDMachine can, but only a small fraction of public data will be available.
The LSST WFD survey observing strategy is yet to be specified.  All surveys considered give complete discovery out to $z=0.3$, but
they vary widely on their ability to yield accurate distances or early classification.  Either way, none of these public data
(nor any of the private) will provide the precision that spectrophotometry can.  For this reason, LBL's primary interest
is in developing a follow-up network.

There are several classes of follow-up we have identified:
\begin{itemize}
\item Spectrophotometry around peak brightness.  Spectrophotometry is expected to give 0.08~mag distance modulus uncertainties compared
to the 0.12~mag uncertainties of a private ZTF or good WFD LSST photometry.  Spectrophotometry increases the power
of a single supernova by $\times 2.25$ while also getting a host redshift.
\begin{itemize}
\item SN Ia Classification -- SNIFS can supplement SED Machine to go toward 100\% SN~Ia classifications
while going deeper than the nominal $18.5$~mag limit of SED Machine.
\item Host-galaxy redshift.
\item SN Ia Distance -- SNIFS has already been used to standardize SNe~Ia magnitudes to $\sigma_M=0.08$~mag.  SNIFS observations will be
designed to obtain this precision.  This SN subset  will have relatively smaller peculiar-velocity uncertainties relative to 
those with only ZTF2 photometry.  The SNIFS IFU provides local host-galaxy properties, which may also improve SN distance precisions.
\end{itemize}
SNIFS at the UH-88 is an existing instrument that could do the job and Greg is communicating with UH about
a robotized upgrade to follow-up ZTF discoveries.
Alex submitted a proposal to instrument the CAHA 2m with
an IFU to follow ZTF discoveries.  French IN2P3 Colleagues are interested in installing a MUSE clone on the VST.
We need to identify opportunities at telescopes that we can instrument with  IFU spectrographs
built by LBL and most likely collaborators in Lyon and Paris who have expertise.

The University of Hawaii must allocate time and resources into the program.  There is already UH expertise in supernovae and peculiar velocities and an existing relationship
with LBL.

\item DESI can provide the following components to supplement surveys:
\begin{itemize}
\item Discovery Screening -- Provides redshifts for probable host galaxies of new transients, for use in early classification.  A host galaxy may
already have a DESI redshift, may be a BGS target without a redshift but whose observation could be prioritized,  or non-DESI target for which we
make a secondary-target fiber allocation.
\item SN Ia Classification -- Spectroscopy of active transients  through secondary-target fiber allocations within
DESI survey pointings. There
will be $\sim 1$ active $r<21.5$ SN~Ia in every three DESI pointings.  Coordinating DESI pointings such that ``every'' pointing contains
a  ZTF2 discovery can significantly increase the number of classifications, relative to the random (from the transient perspective) default DESI pointings.
Triggered observations of non-DESI pointings is possible, though does not take advantage of DESI multiplexing.
\item Host-galaxy redshift -- Mopping up missing host-galaxy redshifts
can be done efficiently with a single sweep of DESI's MOS.
\end{itemize}
 The $R>2000$ resolution
provides sufficiently precise redshifts so as to make their uncertainties negligible in the $\gamma$ error budget.
\item NIR -- Something through UH?
\begin{itemize}
\item SN Ia Distance -- NIR observations are designed to get $\sigma_M=0.08$~mag.  These SNe will be more sensitive probes of velocity
than those with only ZTF2 photometry.
\end{itemize}
\item Optical imaging.  LSST may choose a WFD strategy that while discovering SNe~Ia, provide so little data on them
as to render them useless.  Even if LSST does have a good observing strategy, the closest SNe~Ia that we are interested in
will saturate the LSST Camera.  Depending on the circumstances,
a ``ZTF-South'' instrument capable of SN discovery early in their evolution and not saturating is an option.

Identify 1m-class Schmidts that are available.
\end{itemize}

\subsection{ZTF-II}

ZTF is soon concluding and LBL advocates ZTF-II as the source of SNe for a near-term peculiar-velocity program
for the following reasons:
\begin{itemize}
\item A peculiar-velocity survey is already being touted as a primary science driver. As such,
the observing strategy should accommodate our needs.
\item It would be ready to start Fall 2020.
\item It is cheap, with a cost ranging from zero to access public data, to an amount (\$200k/yr) smaller
than building a new facility.
\item ZTF2 includes SEDMachine for classification of $m<18.5$~mag  transients.
\item It is in the Northern Hemisphere, which complements and is not superseded by LSST. 
Even after the nominal 3-year survey and simultaneous with LSST, the facility remains important for peculiar-velocity studies.
\item It is anticipated that the public plus private collaboration surveys can be designed to generate distance
precisions of $\sigma_M =0.12$~mag, which allow good velocity measurements with SNe~Ia.  (Additional follow-up can lower this uncertainty further.)
\item The limiting redshift $z_{\text{max}} =0.09$ is sufficiently deep  to have a scientifically interesting result $\sigma_\gamma < 0.053$.
There are other SN searches that do not achieve this depth.
\end{itemize}

ZTF2 SN~Ia discoveries are combined with data from other facilities, as described above, to form a complete PV program.  
The active SN spectrophotometric and NIR follow-up provide significant distances estimates over
what ZTF2 photometry can do alone.




ZTF2 will have public, private collaboration, and CalTech surveys.    In ZTF,
the private time was used to survey in the $i$-band (supplementing the $g$ and $r$-bands of the public survey), that turns out to be useful in transient classification and SN~Ia distance determination.
%Nevertheless, we should monitor whether the public data is sufficient for our needs.  It could be that we do not need to buy into ZTF2 in order to have
%a ZTF2-discovery peculiar velocity survey.  (Input from current ZTF folks should be solicited.)

Option A: Do not join the partnership and use ZTF-II public data.
ZTF-II public should provide discoveries and classifications of $z<0.09$ efficiently.  SEDMachine spectra for a
fraction of the bright objects.  Public pixels will be available in two months.
The pro is that the discoveries are free.
Cons are  that to be competitive we nonetheless need to get more external data to make up for the ZTF-II private data
we don't get as ZTF-II two-band photometry will not provide good distances.  
Either way precise redshifts have to come from somewhere.


Private buy-in probably gives slightly more solid-angle, $i$ photometry, and more SEDMachine spectra.  With the third band distance precisions are expected
to be usable for PV, though not as good as with spectra at maximum.  It gives a say as to how to steer partnership time, though there is no guarantee that
you can generate a majority.

The overall assessment is that buy-in would be nice as a second but not first priority.

\subsection{Long Term: LSST +}
The DESC SN~Ia Working Group is interested in peculiar velocity science.  There is an official peculiar-velocity project.  A peculiar-velocity
metric was included in the DESC response to the Project call for white papers on observing strategies.
Informal meetings have been held by DESC members.

The nominal LSST observing strategy will efficiently discover SNe~Ia at $z<0.3$.  However, its poor temporal sampling will not provide accurate
distances nor early classification of SNe~Ia near maximum light.  Our French colleagues have identified a different strategy that will provide
decent sampling that give accurate distances and classifications.  LBL's approach varies depends on whether the Project adopts an acceptable
WDF observing strategy.  A systemic problem is the saturation of nearby SNe around peak brightness.

If it does not, we can either run a ``ZTF South'' that is coordinated with LSST, or the surface density is low enough that we could take
a targeted approach.  The combined photometry can be used to get distances.  This would be supplemented with spectroscopy as described below.

If LSST WFD does get respectable distances and classifications, our primary interest is value added information from spectroscopy.

\subsection{A New Project}
The scope of ZTF2 and the coordination of follow-up of LSST discoveries extend beyond the confines
of current DOE projects.  The recommendations
of the  Small Projects Portfolio  by the  Cosmic Visions Dark Energy Working Group
provides an path by which LBL could lead an international collaboration,
supported by the Office of Science, in the study of Peculiar Velocities.

The new peculiar velocity project would focus on two topics: the use of DESI (and future spectroscopic surveys
such as DESI2) for measuring distances of
fundamental plane galaxies; the mobilization of follow-up resources and data management that are required or enhance 
the probative power of transient discoveries by ZTF2/LSST.  Ideas being discussed for the latter include
refurbishment and use of the
UH-88 + SNIFS, DESI, 4MOST, a proposed French spectrograph mounted at ESO,
a network of identical spectrographs (e.g.\ the DESI design) distributed around the world.
There is expressed interest from South Africa and Australia in using their resources for peculiar-velocity follow-up observations.

LBL, through the project, would support a broad community interested in peculiar velocities. 
There are interested groups at the  University of Hawaii, the University of Michigan, the University of Pittsburgh, the University of Rochester, Yale University,  Brookhaven National Laboratory,
the Carnegie Observatories, several IN2P3 labs, Humboldt University, the University of Toronto,  the University of Queensland, African Institute for Mathematical
Sciences.